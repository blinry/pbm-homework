\documentclass[10pt,DIV10,a4paper]{scrartcl}
\usepackage[utf8]{inputenc}
\usepackage[german]{babel}
\usepackage{amsmath}
\usepackage{amsfonts}
\usepackage{amssymb}
\usepackage{microtype}

\title{PMS -- Exercise Sheet 8}
\date{}

\begin{document}

\maketitle

\section*{Exercise 1}

\subsection*{Initialisierung}

\(A=\left(
\begin{array}{ccc}
 1 & 1 & 4 \\
 1 & 1 & 1 \\
 4 & 1 & 1 \\
\end{array}
\right)\)

\(b=\left(
\begin{array}{c}
 6 \\
 6 \\
 6 \\
\end{array}
\right)\)

\(x_0=\left(
\begin{array}{c}
 0 \\
 0 \\
 0 \\
\end{array}
\right)\)

\subsection*{Erste Iteration}

\(d_0=b - A * x_0 = \left(
\begin{array}{c}
 6 \\
 6 \\
 6 \\
\end{array}
\right)-\left(
\begin{array}{ccc}
 1 & 1 & 4 \\
 1 & 1 & 1 \\
 4 & 1 & 1 \\
\end{array}
\right)*\left(
\begin{array}{c}
 0 \\
 0 \\
 0 \\
\end{array}
\right)=\left(
\begin{array}{c}
 6 \\
 6 \\
 6 \\
\end{array}
\right)\)

\(r_0=d_0=\left(
\begin{array}{c}
 6 \\
 6 \\
 6 \\
\end{array}
\right)\)

\(\alpha _0=\frac{r_0{}^T* r_0}{d_0{}^T*A *d_0}=\frac{\left(
\begin{array}{c}
 6 \\
 6 \\
 6 \\
\end{array}
\right)^T* \left(
\begin{array}{c}
 6 \\
 6 \\
 6 \\
\end{array}
\right)}{\left(
\begin{array}{c}
 6 \\
 6 \\
 6 \\
\end{array}
\right)^T* \left(
\begin{array}{ccc}
 1 & 1 & 4 \\
 1 & 1 & 1 \\
 4 & 1 & 1 \\
\end{array}
\right)*\left(
\begin{array}{c}
 6 \\
 6 \\
 6 \\
\end{array}
\right) }=\frac{1}{5}\)

\(x_1= x_0+\alpha _0d_0= \left(
\begin{array}{c}
 0 \\
 0 \\
 0 \\
\end{array}
\right) + \frac{1}{5}\left(
\begin{array}{c}
 6 \\
 6 \\
 6 \\
\end{array}
\right) = \left(
\begin{array}{c}
 \frac{6}{5} \\
 \frac{6}{5} \\
 \frac{6}{5} \\
\end{array}
\right)\)

\subsection*{Zweite Iteration}

\(r_1=r_0- \alpha _0 A * d_0=\left(
\begin{array}{c}
 6 \\
 6 \\
 6 \\
\end{array}
\right)-\frac{1}{5} \left(
\begin{array}{ccc}
 1 & 1 & 4 \\
 1 & 1 & 1 \\
 4 & 1 & 1 \\
\end{array}
\right) *\left(
\begin{array}{c}
 6 \\
 6 \\
 6 \\
\end{array}
\right) = \left(
\begin{array}{c}
 -\frac{6}{5} \\
 \frac{12}{5} \\
 -\frac{6}{5} \\
\end{array}
\right)\)

\(\beta _1= \frac{r_1{}^Tr_1}{r_0{}^Tr_0}=\frac{\left(
\begin{array}{c}
 -\frac{6}{5} \\
 \frac{12}{5} \\
 -\frac{6}{5} \\
\end{array}
\right)^T* \left(
\begin{array}{c}
 -\frac{6}{5} \\
 \frac{12}{5} \\
 -\frac{6}{5} \\
\end{array}
\right)}{\left(
\begin{array}{c}
 6 \\
 6 \\
 6 \\
\end{array}
\right)^T* \left(
\begin{array}{c}
 6 \\
 6 \\
 6 \\
\end{array}
\right)}=\frac{2}{25}\)

\(d_1= r_1+\beta _1d_0=  \left(
\begin{array}{c}
 -\frac{6}{5} \\
 \frac{12}{5} \\
 -\frac{6}{5} \\
\end{array}
\right) + \frac{2}{25}\left(
\begin{array}{c}
 6 \\
 6 \\
 6 \\
\end{array}
\right)=\left(
\begin{array}{c}
 -\frac{18}{25} \\
 \frac{72}{25} \\
 -\frac{18}{25} \\
\end{array}
\right)\)

\(\alpha _1=\frac{r_1{}^T* r_1}{d_1{}^T* A * d_1}=\frac{\left(
\begin{array}{c}
 -\frac{6}{5} \\
 \frac{12}{5} \\
 -\frac{6}{5} \\
\end{array}
\right)^T* \left(
\begin{array}{c}
 -\frac{6}{5} \\
 \frac{12}{5} \\
 -\frac{6}{5} \\
\end{array}
\right)}{\left(
\begin{array}{c}
 -\frac{18}{25} \\
 \frac{72}{25} \\
 -\frac{18}{25} \\
\end{array}
\right)^T* \left(
\begin{array}{ccc}
 1 & 1 & 4 \\
 1 & 1 & 1 \\
 4 & 1 & 1 \\
\end{array}
\right) *\left(
\begin{array}{c}
 -\frac{18}{25} \\
 \frac{72}{25} \\
 -\frac{18}{25} \\
\end{array}
\right)}= \frac{5}{3}\)

\(x_2=x_1+\alpha _1d_1= \left(
\begin{array}{c}
 \frac{6}{5} \\
 \frac{6}{5} \\
 \frac{6}{5} \\
\end{array}
\right) + \frac{5}{3}\left(
\begin{array}{c}
 -\frac{18}{25} \\
 \frac{72}{25} \\
 -\frac{18}{25} \\
\end{array}
\right)=\left(
\begin{array}{c}
 0 \\
 6 \\
 0 \\
\end{array}
\right)\text{}\)

\(r_2=r_1-\alpha _1 A * d_1= \left(
\begin{array}{c}
 -\frac{6}{5} \\
 \frac{12}{5} \\
 -\frac{6}{5} \\
\end{array}
\right) - \frac{5}{3}\left(
\begin{array}{ccc}
 1 & 1 & 4 \\
 1 & 1 & 1 \\
 4 & 1 & 1 \\
\end{array}
\right) * \left(
\begin{array}{c}
 -\frac{18}{25} \\
 \frac{72}{25} \\
 -\frac{18}{25} \\
\end{array}
\right)=\left(
\begin{array}{c}
 0 \\
 0 \\
 0 \\
\end{array}
\right)\)

Abbruch, da Residuum \(r_2=0 \Rightarrow\) keine weitere {\" A}nderung an gefundenem Optimum \(x_2=\left(
\begin{array}{c}
 0 \\
 6 \\
 0 \\
\end{array}
\right)\)

\section*{Exercise 2}

\begin{itemize}
    \item[a)] As each step of the algorithm finds the local minimum of one dimension, for a 3-dimensional systems it will at most take three iterations.
    \item[b)] In large systems, the conjugate gradient algorithm allows for fast approximation, doing only a few iterations until the solution is “good enough”. Gauss elimination does not allow this. TODO: More?
    \item[c)] $A$ has to be a symmetric, positive-definite matrix. All symmetric matrices are also square.
    \item[d)] TODO
\end{itemize}

\end{document}
